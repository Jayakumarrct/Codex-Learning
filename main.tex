\documentclass[12pt]{article}
\usepackage{amsmath,amsthm,amssymb,hyperref}
\newtheorem{theorem}{Theorem}[section]
\newtheorem{lemma}[theorem]{Lemma}
\newtheorem{corollary}[theorem]{Corollary}
\theoremstyle{definition}
\newtheorem{definition}[theorem]{Definition}
\newtheorem{remark}[theorem]{Remark}
\newtheorem{example}[theorem]{Example}

\title{Lecture Notes}
\author{Jay}
\date{\today}

\begin{document}
\maketitle
\tableofcontents

\section{Warm-up}\label{sec:warmup}

\begin{theorem}\label{thm:identity}
If $G$ is a group, the identity element is unique.
\end{theorem}

\begin{proof}
Standard argument: if $e,e'$ are identities, then $e=e\cdot e'=e'$.
\end{proof}

\begin{lemma}\label{lem:inv-unique}
In a group, inverses are unique.
\end{lemma}

\begin{proof}
If $g^{-1}$ and $h$ are both inverses of $g$, then $g^{-1}=g^{-1}gh=h$.
\end{proof}

\begin{definition}\label{def:group}
A group is a set $G$ with an operation $\cdot$ satisfying associativity, identity, and inverses.
\end{definition}

\begin{remark}\label{rem:group-notation}
Group notation often suppresses the operation symbol.
\end{remark}

\begin{example}\label{ex:integers}
The integers $\mathbb{Z}$ with addition form a group.
\end{example}

\begin{corollary}\label{cor:inv-inv}
For $g\in G$, the inverse of $g^{-1}$ is $g$.
\end{corollary}

\begin{proof}
By Lemma~\ref{lem:inv-unique}, $g$ is the unique inverse of $g^{-1}$.
\end{proof}
\end{document}
